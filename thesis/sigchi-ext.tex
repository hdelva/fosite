\documentclass[8pt,a4paper,twocolumn]{article}

%%%%%%%%%%%%%%%%%%%%%%%%%%%%%%%%%%%%%%%%%%%%%%%%%%%%%%%%%%%%%%%%%%%%%%
% Document preamble
%%%%%%%%%%%%%%%%%%%%%%%%%%%%%%%%%%%%%%%%%%%%%%%%%%%%%%%%%%%%%%%%%%%%%%

%% Builds upon the graphics  package, providing a key-value interface
%% for optional arguments to the \includegraphics command that go far
%% beyone what the graphics package offers.
%% http://www.ctan.org/tex-archive/help/Catalogue/entries/graphicx.html
%% if you use PostScript figures in your article
%% use the graphics package for simple commands
%% \usepackage{graphics}
%% or use the graphicx package for more complicated commands
%% \usepackage{graphicx}
%% or use the epsfig package if you prefer to use the old commands
%% \usepackage{epsfig}
\usepackage{graphicx} % Enhanced LaTeX Graphics
\usepackage{siunitx}

%Tipo de letra Arial
\usepackage{helvet}
\renewcommand{\familydefault}{\sfdefault}

% acentos e cedilhas
\usepackage[utf8]{inputenc}
%\usepackage[T1]{fontenc}

% Multiple figures
%\usepackage{subfigure} % subcaptions for subfigures
%\usepackage{subfigmat} % matrices of similar subfigures

\usepackage[font=footnotesize, skip = 1pt, labelfont=bf]{caption}
\usepackage[font=footnotesize]{subcaption}

% Declaring new column types
% 'dcolumn' package defines D to be a column specifier with
% three arguments: D{<sep.tex>}{<sep.dvi>}{<decimal places>}
%                  D{<sep.tex>}{<sep.dvi>}{<left digit places>.<right digit places>}
\usepackage{dcolumn}           % decimal-aligned tabular math columns
% d takes a single argument specifying the number of decimal places, e.g., d{2}
% or the number of digits to the left and right of the seperator, e.g., d{3.2}
\newcolumntype{.}   {D{.}{.}{-1}} % column alignedd on the point separator '.'
\newcolumntype{d}[1]{D{.}{.}{#1}} % column centered on the point separator '.'
\newcolumntype{e}   {D{E}{E}{-1}} % column centered on the exponent 'E'
\newcolumntype{E}[1]{D{E}{E}{#1}} % column centered on the exponent 'E'

%% American Mathematical Society (AMS) plain Tex macros
%%
%% The amsmath package is the principal package in the AMS-LaTeX distribution
%% http://www.ctan.org/tex-archive/help/Catalogue/entries/amsmath.html
\usepackage{amsmath}
\DeclareMathSizes{7}{7}{3}{3} 
\usepackage{pifont}
%%
%% The amsfonts package provides extended TeX fonts
%% http://www.ctan.org/tex-archive/help/Catalogue/entries/amsfonts.html
\usepackage{amsfonts}
%% The amssymb package provides various useful mathematical symbols
\usepackage{amssymb}
%%
%% The amsthm package provides extended theorem environments
%% http://www.ctan.org/tex-archive/help/Catalogue/entries/amsthm.html
\usepackage{amsthm}

%% Improves the interface for defining floating objects such as figures and tables.
%% The package also provides the H float modifier option of the obsolete here package.
%% http://www.ctan.org/tex-archive/help/Catalogue/entries/float.html
\usepackage{float}

%% Control sectional headers. 
%% http://www.ctan.org/tex-archive/help/Catalogue/entries/sectsty.html
\usepackage{sectsty}
%%
%% Redefine the font size of the 'section' and 'subsection' headings
\newcommand{\myFontSize}{\fontsize{9}{0}\selectfont}
\sectionfont{\myFontSize}       % 10pt, Bold face (default)
\subsectionfont{\myFontSize} % 10pt, Plain face

%% Select alternative section titles.
%% http://www.ctan.org/tex-archive/help/Catalogue/entries/titlesec.html
\usepackage{titlesec}
\usepackage{booktabs}
%\usepackage{multirow}
%\usepackage{array}
\usepackage{csquotes}% Recommended
%\usepackage[style=authoryear, backend=bibtex, doi=false,isbn=false,url=false,eprint=false,dashed=false,maxcitenames=2, maxbibnames=100]{biblatex}
%\addbibresource{library.bib}

%%
%% Left indent, before and after spacing
%% (The starred version kills the indentation of the paragraph following the title)
\titlespacing*{\section}{0pt}{10pt}{0pt}
\titlespacing*{\subsection}{0pt}{10pt}{0pt}

%% Section numbers with trailing dots. 
%% http://www.ctan.org/tex-archive/help/Catalogue/entries/secdot.html
\usepackage{secdot}
\usepackage{epstopdf}
%%
%% Also put a dot after the subsection number
\sectiondot{subsection}
%% Set a space between dot and heading text
\sectionpunct{section}{. }    % By default, \sectiondot places a \quad
\sectionpunct{subsection}{. } % after the number

% These are exact settings for a A4 page with top margin of
% 25 mm, bottom margin of 30 mm, left and right margins of 25 mm,
% printable area 242 X 160 mm.

\setlength{\topmargin}{-10.4mm}
\setlength{\headheight}{0.0mm}
\setlength{\headsep}{10.0mm}
\setlength{\textwidth}{160mm}
\setlength{\textheight}{242mm}
\setlength{\oddsidemargin}{0mm}
\setlength{\evensidemargin}{0mm}
\setlength{\marginparwidth}{0mm}
\setlength{\marginparsep}{0mm}
\usepackage{multicol}

% New command to refer to equations as Eq.(1),Eq.(2),...
\newcommand{\eqnref}[1]{Eq.(\ref{#1})}

%%%%%%%%%%%%%%%%%%%%%%%%%%%%%%%%%%%%%%%%%%%%%%%%%%%%%%%%%%%%%%%%%%%%%%%%%%%%%%%%%%%%%%%%
% Title, authors and addresses

\title{\bfseries Thesis Title}
\date{April 2016}
\author{Author name \\ author.name@tecnico.ulisboa.pt \\ \\ Instituto Superior Técnico, Lisboa, Portugal}

%%%%%%%%%%%%%%%%%%%%%%%%%%%%%%%%%%%%%%%%%%%%%%%%%%%%%%%%%%%%%%%%%%%%%%%%%%%%%%%%%%%%%%%%
\begin{document}

% Begin one column section for title and abstract
%
% http://www.faqs.org/faqs/de-tex-faq/part5/

\maketitle

\begin{abstract}
fucking kill me ouhds sdih iudh uhdsviudhsh osvhuhdsv
hue

hue

\end{abstract}

\begin{multicols}{2}
\section{Introduction}
negers negers negers negers negers negers negers negers negers negers negers negers negers negers negers negers negers negers negers negers negers negers negers negers negers negers negers negers negers negers negers negers negers negers negers negers negers negers negers negers negers negers negers negers negers negers negers negers negers negers negers negers 
\end{multicols}





% REFERENCES

% Produces the bibliography section when processed by BibTeX
%
% Bibliography style
% > entries ordered alphabetically
%\bibliographystyle{plain}
% > unsorted with entries appearing in the order in which the citations appear.
%\bibliographystyle{unsrt}
% > entries ordered alphabetically, with first names and names of journals and months abbreviated
\bibliographystyle{abbrv}
% > entries ordered alphabetically, with reference markers based on authors' initials and publication year
%\bibliographystyle{alpha}

% External bibliography database file in the BibTeX format (ExtendedAbstract_ref_db.bib)
\bibliography{references}

\end{document}



              
            
Start with our Templates

Overleaf is perfect for all types of projects — from papers and presentations to newsletters, CVs and much more! It's also a great way to learn how to use LaTeX and produce professional looking projects quickly.

Make your Own

Upload or create templates for journals you submit to and theses and presentation templates for your institution. Just create it as a project on Overleaf and use the publish menu. It's free! No sign-up required.

Follow us for More

New template are added all the time. Follow us on twitter for the highlights!

Overleaf is a free online collaborative LaTeX editor. No sign up required.
Learn more
Search for more Templates, Articles and Examples

Latest News
May 18, 2017
Wiley partnership with Overleaf enables author collaboration
May 18, 2017
Guest Post Feature: How to Promote Consistency in Collaborative Writing
Overleaf
Getting Started
Templates & Gallery
iPad & Tablet
Plans & Pricing
Referrals
Teams
Support
Help & FAQs
Privacy & Terms
Security Overview
Developers
Company
About us
Contact us
Partners
Advisors
Press
Blog
Jobs
   
Writelatex Limited, 3rd Floor, 207 Regent Street, London, W1B 3HH, UKCopyright © 2017. All rights reserved.